\documentclass{article}
\usepackage{titlesec}
\usepackage{titling}
\usepackage[margin=.5in]{geometry}
\usepackage{enumitem}

\setlist{nosep}

\titleformat{\section}
{\bfseries\normalsize}
{}
{0em}
{}[\titlerule]

\titleformat{\subsection}[runin]
{\bfseries}
{}
{0em}
{}[:]

\titleformat{\subsubsection}[runin]
{}
{$\bullet$}
{0em}
{}

\titlespacing{\subsection}
{0em}{0em}{1em}

\titlespacing{\subsubsection}
{3em}{0em}{1em}

\renewcommand{\maketitle}{
	\begin{center}
		{\huge\bfseries
			\theauthor}
			
		404.430.1346 $\bullet$ cacost12@asu.edu $\bullet$ US Citizen $\bullet$ linkedin.com/in/colton-acosta/
	\end{center}
}

\renewcommand{\baselinestretch}{1.1}

\begin{document}

\title{Resume}
\author{Colton Acosta}
\maketitle
\section{SUMMARY}
Junior aerospace engineering student with collaborative project experience in hardware-software interfacing, specification development, computer-aided modeling, and systems integration. Interests include control theory, signal processing, and avionics. Open to relocation. 
\section{EDUCATION}
\textbf{B.S.E, Aerospace Engineering;} Autonomous Vehicle Systems
\hfill 
Graduating May 2022
\linebreak
Arizona State University, Tempe, AZ 
\hfill
4.00 GPA
\linebreak
Barrett, The Honors College

\section{TECHNICAL SKILLS}
\subsection{Presentation and Organization}
Microsoft Office,  {\LaTeX}
\subsection{Design and Modeling}
MATLAB, SOLIDWORKS, LabVIEW
\subsection{Programming} 
Python, C/C++, Linux (git, vim, gcc), R
\section{EXPERIENCE}
\textbf{Liquid Propulsion Avionics Lead, Sun Devil Rocketry}
\hfill
\vspace{0.5em}
August 2019-Present
\begin{itemize}
	\item{Leading efforts on the development of an avionics system for a liquid bi-propellant engine with over twenty hardware components including valves, sensors, and controllers}
	\item{Interfacing temperature, pressure, thrust, and flow measurements with a custom Python program}
	\item{Conducting trade studies on electronic actuators and orifice flow meters to develop specification requirements using MATLAB for multivariate trade-off analyses and physical modeling}
	\item{Built a signal amplifier to boost sensor outputs to measurable ranges resulting in savings upwards of \$200}
	\item{Reported project progress in published AIAA Propulsion and Energy conference paper}
	\item{Wrote a development plan for the 2020-2021 academic year consisting of 42 deliverables to cultivate a results-oriented work environment and to delegate workloads among new talent}
\end{itemize}
\vspace{1em}
\textbf{5280 Team Member, Sun Devil Rocketry}
\hfill Fall 2018-Spring 2019
\vspace{0.5em}
\begin{itemize}
	\item Collaborated with a group of 12 students to launch an amateur rocket to an altitude of 5280 feet
	\item Determined build specifications, apogee altitude, and static margin with OpenRocket software
	\item Constructed rocket with phenolic tubing wrapped with epoxied fiberglass fabric and laser-cut fins 
	\item Pragmatically adjusted final build to mitigate static margin error, allowing rocket to be launched on time
	\item Used a microcontroller breakout board with internal altimeter for parachute deployment
\end{itemize}
\section{ACADEMIC PROJECTS}

\textbf{Team Lead, Airfoil Statistics Project}\hfill Spring 2020
\vspace{0.5em}
\begin{itemize}
	\item Lead a group of seven students to complete a semester long project by devising a project plan, delegating workloads, setting timelines, and scheduling team meetings
	\item Collected computational aerodynamic data with Ansys for airflow over a wing section with configurable design specifications created with SOLIDWORKS from a NACA 2412 airfoil parameterization
	\item Used the R language to analyze data with an Analysis of Variance test and to compute significance-based regression models for wing section lift and drag response to chord length, angle of attack, and sweep angle
\end{itemize}
\vspace{0.5em}
\textbf{Honors Student, Trans-lunar Injection Simulation} 
\hfill
Spring 2019
\vspace{0.5em}
\begin{itemize}
	\item Simulated a free-return, trans-lunar injection orbital trajectory in MATLAB with an animated solution
	\item Calculated the trajectory by solving the two-body problem using a numerical differential equation solver built from scratch with Apollo 11 low earth orbit initial conditions
\end{itemize}

\thispagestyle{empty}
\end{document}