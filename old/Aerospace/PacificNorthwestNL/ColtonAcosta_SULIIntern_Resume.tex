\documentclass{article}
\usepackage{titlesec}
\usepackage{titling}
\usepackage[margin=.5in]{geometry}
\usepackage{enumitem}

\setlist{nosep}

\titleformat{\section}
{\bfseries\normalsize}
{}
{0em}
{}[\titlerule]

\titleformat{\subsection}[runin]
{\bfseries}
{}
{0em}
{}[:]

\titleformat{\subsubsection}[runin]
{}
{$\bullet$}
{0em}
{}

\titlespacing{\subsection}
{0em}{0em}{1em}

\titlespacing{\subsubsection}
{3em}{0em}{1em}

\renewcommand{\maketitle}{
	\begin{center}
		{\huge\bfseries
			\theauthor}
			
		404.430.1346 $\bullet$ cacost12@asu.edu $\bullet$ US Citizen $\bullet$ linkedin.com/in/colton-acosta/
	\end{center}
}

\renewcommand{\baselinestretch}{1.0}

\begin{document}

\title{Resume}
\author{Colton Acosta}
\maketitle
\section{SUMMARY}
Junior aerospace engineering student with leadership and collaborative project experience including electronics, hardware-software interfacing, computer-aided design and modeling, and programming. Interests include control theory, signal processing, avionics, and propulsion. Open to relocation. 
\section{EDUCATION}
\textbf{B.S.E, Aerospace Engineering;} Autonomous Vehicle Systems
\hfill 
Graduating May 2022
\linebreak
Arizona State University, Tempe, AZ 
\hfill
4.00 GPA

\section{TECHNICAL SKILLS}
\subsection{Presentation and Organization}
Microsoft Office,  {\LaTeX}
\subsection{Design and Modeling}
MATLAB/Simulink, SOLIDWORKS, LabVIEW
\subsection{Programming} 
Python, C/C++, Linux (git, vim, gcc)
\section{EXPERIENCE}
\textbf{Liquid Propulsion Avionics Lead, Sun Devil Rocketry}
\hfill
\vspace{0.5em}
August 2019-Present
\begin{itemize}
	\item{Leading the design and development of an integrated avionics system for a liquid rocket engine with over 20 hardware components including valves, sensors, controllers, and signal processing circuitry}
	\item{Interfacing temperature, pressure, thrust, and flow measurements with a Python graphical user interface}
	\item{Built an instrumentation amplifier circuit using operational amplifiers to boost sensor outputs to measurable ranges resulting in hardware savings upwards of \$200}
	\item{Designed and built a second order, active low-pass filter and tested the filter's noise reduction and signal reproduction by adding noise to a measured signal with a voltage summing circuit}
	\item{Programming Arduino controllers for prototype testing of data acquisition and flow control embedded systems}
	\item{Conducting trade studies on electronic actuators and orifice flow meters to develop main propulsion system specifications using MATLAB for multivariate trade-off analyses and physical modeling}
	\item{Documented project progress in published AIAA Propulsion and Energy conference paper}
	\item{Wrote a development plan for the 2020-2021 academic year consisting of 42 deliverables to document project milestones, cultivate a results-oriented work environment, and delegate workloads among new talent}	
\end{itemize}
\section{PROJECTS}
\textbf{Flow Control Valve Actuator Control System}\hfill Fall 2020
\vspace{0.5em}
\begin{itemize}
	\item Designed and built a closed loop control system for valve actuation in flow throttling applications
	\item Examined the relationship between pulse width modulation duty cycle and steady state shaft speed to derive a controller output signal with a linear transfer function from controller output to shaft position  
	\item Characterized the plant transfer function with a series of step response experiments
	\item  Designed and simulated a saturated PI controller with integrator clamping using Simulink with performance specifications of zero steady state error of step inputs and complete rejection of step disturbances.
	\item  Built the actuator control system using a brushed DC motor, coupling shaft, Arduino controller, and quadrature rotary encoder for feedback.
\end{itemize}
\vspace{0.5em}
\textbf{5280 Team Member, Sun Devil Rocketry}
\hfill Fall 2018-Spring 2019
\vspace{0.5em}
\begin{itemize}
	\item Collaborated with a group of 12 students to launch an amateur rocket to an altitude of 5280 feet
	\item Determined build specifications, apogee altitude, and static margin with OpenRocket software
	\item Constructed rocket with phenolic tubing wrapped with epoxied fiberglass fabric and laser-cut fins 
	\item Used a microcontroller breakout board with internal altimeter for parachute deployment
\end{itemize}
\vspace{0.5em}
\textbf{Orbital Mechanics Trans-lunar Injection Simulation} 
\hfill
Spring 2019
\vspace{0.5em}
\begin{itemize}
	\item Simulated a free-return, trans-lunar injection orbital trajectory in MATLAB with an animated solution
	\item Calculated the trajectory by solving the two-body problem using a numerical differential equation solver built from scratch with Apollo 11 low earth orbit initial conditions
\end{itemize}

\thispagestyle{empty}
\end{document}