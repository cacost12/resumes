\documentclass{article}
\usepackage{titlesec}
\usepackage{titling}
\usepackage[margin=.5in]{geometry}
\usepackage{enumitem}

\setlist{nosep}

\titleformat{\section}
{\bfseries\normalsize}
{}
{0em}
{}[\titlerule]

\titleformat{\subsection}[runin]
{\bfseries}
{}
{0em}
{}[:]

\titleformat{\subsubsection}[runin]
{}
{$\bullet$}
{0em}
{}

\titlespacing{\subsection}
{0em}{0em}{1em}

\titlespacing{\subsubsection}
{3em}{0em}{1em}

\renewcommand{\maketitle}{
	\begin{center}
		{\huge\bfseries
			\theauthor}
			
		404.430.1346 $\bullet$ cacost12@asu.edu $\bullet$ US Citizen $\bullet$ linkedin.com/in/colton-acosta/
	\end{center}
}

\renewcommand{\baselinestretch}{1.0}

\begin{document}

\title{Resume}
\author{Colton Acosta}
\maketitle
\section{SUMMARY}
Junior aerospace engineering student with collaborative, multidisciplinary project experience including electronic systems, hardware-software interfacing, computer-aided modeling, and programming. Interests include communications systems, electromechanical systems, and avionics. Open to relocation. 
\section{EDUCATION}
\textbf{B.S.E, Aerospace Engineering;} Autonomous Vehicle Systems
\hfill 
Graduating May 2022
\linebreak
Arizona State University, Tempe, AZ 
\hfill
4.00 GPA

\section{TECHNICAL SKILLS}
\subsection{Design and Modeling}
MATLAB/Simulink, SOLIDWORKS, Microsoft Office, LabVIEW
\subsection{Hardware}
Soldering, Digital Multimeters, Oscilloscope, Operational Amplifiers, Microcontrollers
\subsection{Programming} 
Python, C, C++, Linux (git, vim, gcc, gdb)
\section{EXPERIENCE}
\textbf{Liquid Propulsion Avionics Lead, Sun Devil Rocketry}
\hfill
\vspace{0.5em}
August 2019-Present
\begin{itemize}
	\item{Leading the design and development of an integrated avionics system for a liquid rocket engine with over 20 hardware components including valves, motors, sensors, microcontrollers, and signal processing circuitry}
	\item{Built an instrumentation amplifier circuit using operational amplifiers to boost sensor outputs to measurable ranges resulting in hardware savings upwards of \$200}
	\item{Designed and built a second order, active low-pass filter and tested the filter's noise reduction and signal reproduction by adding noise to a measured signal with a voltage summing circuit}
	\item{Building a central telemetry system using RS-485 electrical interfaces for long distance and noise insensitive serial communications between data acquisition, valve control, and main controllers}
	\item{Programming Arduino controllers with C++ for prototype testing of the engine's embedded systems including data acquisition, actuation, flow control, and communications functionality}
	\item{Wrote a C program to encode the state of the engine's valves using bit operators for efficient serial data transmission}
	\item{Interfacing temperature, pressure, thrust, and flow measurements with a Python graphical user interface}
	\item{Conducting trade studies on electronic actuators and orifice flow meters to develop main propulsion system specifications using MATLAB for multivariate trade-off analyses and physical modeling}
	\item{Wrote a C++ program to generate Gaussian noise for hardware filter testing by writing an algorithm for computing values of an inverse Gaussian cumulative distribution function}
	\item{Documented project progress in published AIAA Propulsion and Energy conference paper}
	\item{Wrote a development plan for the 2020-2021 academic year consisting of 42 deliverables to document project milestones, cultivate a results-oriented work environment, and delegate workloads among new talent}
\end{itemize}
\section{PROJECTS}
\textbf{Electronic Valve Actuator Control System}\hfill Fall 2020
\vspace{0.5em}
\begin{itemize}
	\item Designed and built a closed loop control system for a valve actuator for use in flow throttling applications
	\item Examined the relationship between Pulse Width Modulation duty cycle and steady state shaft speed to derive a controller output signal with a linear transfer function from controller output to shaft position  
	\item Characterized the plant transfer function a series of step response experiments
	\item  Designed and simulated a saturated PI controller with integrator clamping using Simulink with performance specifications of zero steady state error of step inputs and complete rejection of step disturbances.
	\item  Built the actuator control system using a brushed DC motor, coupling shaft, Arduino controller, and quadrature rotary encoder for feedback.
\end{itemize}
\vspace{0.5em}
\thispagestyle{empty}
\end{document}