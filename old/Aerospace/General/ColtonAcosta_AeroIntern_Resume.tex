\documentclass{article}
\usepackage{titlesec}
\usepackage{titling}
\usepackage[margin=.5in]{geometry}
\usepackage{enumitem}

\setlist{nosep}

\titleformat{\section}
{\bfseries\normalsize}
{}
{0em}
{}[\titlerule]

\titleformat{\subsection}[runin]
{\bfseries}
{}
{0em}
{}[:]

\titleformat{\subsubsection}[runin]
{}
{$\bullet$}
{0em}
{}

\titlespacing{\subsection}
{0em}{0em}{1em}

\titlespacing{\subsubsection}
{3em}{0em}{1em}

\renewcommand{\maketitle}{
	\begin{center}
		{\huge\bfseries
			\theauthor}
			
		404.430.1346 $\bullet$ cacost12@asu.edu $\bullet$ US Citizen $\bullet$ linkedin.com/in/colton-acosta/
	\end{center}
}

\renewcommand{\baselinestretch}{1.0}

\begin{document}

\title{Resume}
\author{Colton Acosta}
\maketitle
\section{SUMMARY}
Junior aerospace engineering student with leadership and collaborative project experience including work in computer-aided design and modeling, specification development, electronics, hardware-software interfacing, and programming. Interests include avionics, propulsion,  control theory, and signal processing. Open to relocation.
\section{EDUCATION}
\textbf{B.S.E, Aerospace Engineering;} Autonomous Vehicle Systems
\hfill 
Graduating May 2022
\linebreak
Arizona State University, Tempe, AZ 
\hfill
4.00 GPA

\section{TECHNICAL SKILLS}
\subsection{Presentation and Organization}
Microsoft Office,  {\LaTeX}
\subsection{Design and Modeling}
MATLAB, Simulink, SOLIDWORKS, LabVIEW
\subsection{Programming} 
Python, C, C++, Linux (git, vim, gcc, gdb), R
\section{EXPERIENCE}
\textbf{Liquid Propulsion Avionics Lead, Sun Devil Rocketry}
\hfill
\vspace{0.5em}
August 2019-Present
\begin{itemize}
	\item{Leading the design and development of an avionics system for a liquid rocket engine with over 20 hardware components including valves, transducers, thermocouples, load cells, controllers, and signal processing circuitry}
	\item{Defined avionics systems functional requirements with a system architecture diagram}
	\item{Conducting trade studies on electronic actuators and orifice flow meters to develop main propulsion system specifications using MATLAB for multivariate trade-off analyses and physical modeling}
	\item{Interfacing temperature, pressure, thrust, and flow measurements with a Python graphical user interface}
	\item{Built an instrumentation amplifier circuit using operational amplifiers to boost sensor outputs to measurable ranges resulting in hardware savings upwards of \$200}
	\item{Building a central telemetry system using RS-485 electrical interfaces for long distance and noise insensitive serial communications between data acquisition, valve control, and main controllers}
	\item{Designed and built a second order, active low-pass filter and tested the filter's noise reduction and signal reproduction by adding noise to a measured signal with a voltage summing circuit}
	\item{Programming AVR microcontrollers with C for prototype testing of the engine's embedded systems including data acquisition, actuation, flow control, and communications functionality}
	\item{Wrote a C++ program to generate Gaussian noise for hardware filter testing by writing an algorithm for computing values of an inverse Gaussian cumulative distribution function}
	\item{Wrote a C program to encode the state of the engine's valves using bit operators for efficient serial data transmission}
	\item{Wrote, compiled, and debugged all C and C++ code using Linux command line tools such as gcc, g++, gdb, and vim}
	\item{Documented project progress in published AIAA Propulsion and Energy conference paper}
	\item{Wrote a development plan for the 2020-2021 academic year consisting of 42 deliverables to document project milestones, cultivate a results-oriented work environment, and delegate workloads among new talent}
\end{itemize}
\section{PROJECTS}
\textbf{5280 Team Member, Sun Devil Rocketry}
\hfill Fall 2018-Spring 2019
\vspace{0.5em}
\begin{itemize}
	\item Collaborated with a group of 12 students to launch an amateur rocket to an altitude of 5280 feet
	\item Determined build specifications, apogee altitude, and static margin with OpenRocket software
	\item Constructed rocket with phenolic tubing wrapped with epoxied fiberglass fabric and laser-cut fins 
	\item Adjusted final build to pragmatically mitigate static margin calculation error, allowing rocket to be launched on time
	\item Used a microcontroller breakout board with internal altimeter for parachute deployment
\end{itemize}
\vspace{0.5em}
\textbf{Flow Control Valve Actuator Control System}\hfill Fall 2020
\vspace{0.5em}
\begin{itemize}
	\item Designed and built a closed loop control system for a valve actuator for use in flow throttling applications
	\item Examined the relationship between Pulse Width Modulation duty cycle and steady state shaft speed to derive a controller output signal with a linear transfer function from controller output to shaft position  
	\item Characterized the plant transfer function a series of step response experiments
	\item  Designed and simulated a saturated PI controller with integrator clamping using Simulink with performance specifications of zero steady state error of step inputs and complete rejection of step disturbances.
	\item  Built the actuator control system using a brushed DC motor, coupling shaft, Arduino controller, and quadrature rotary encoder for feedback.
\end{itemize}
\vspace{0.5em}
\textbf{Team Lead, Airfoil Statistics Project}\hfill Spring 2020
\vspace{0.5em}
\begin{itemize}
	\item Lead a group of seven students to complete a semester long project by devising a project plan, delegating workloads, setting timelines, and scheduling team meetings
	\item Collected computational aerodynamic data with Ansys for airflow over a wing section with configurable design specifications created with SOLIDWORKS from a NACA 2412 airfoil parameterization
	\item Used R programming to analyze data with an Analysis of Variance test and to compute significance-based regression models for wing section lift and drag response to chord length, angle of attack, and sweep angle factors
\end{itemize}
\vspace{0.5em}
\textbf{Orbital Mechanics Trans-lunar Injection Simulation} 
\hfill
Spring 2019
\vspace{0.5em}
\begin{itemize}
	\item Simulated a free-return, trans-lunar injection orbital trajectory in MATLAB with an animated solution
	\item Calculated the trajectory by solving the two-body problem using a numerical differential equation solver built from scratch with Apollo 11 low earth orbit initial conditions
\end{itemize}

\thispagestyle{empty}
\end{document}