\documentclass{article}
\usepackage{titlesec}
\usepackage{titling}
\usepackage[margin=.5in]{geometry}
\usepackage{enumitem}

\setlist{nosep}

\titleformat{\section}
{\bfseries\normalsize}
{}
{0em}
{}[\titlerule]

\titleformat{\subsection}[runin]
{\bfseries}
{}
{0em}
{}[:]

\titleformat{\subsubsection}[runin]
{}
{$\bullet$}
{0em}
{}

\titlespacing{\subsection}
{0em}{0em}{1em}

\titlespacing{\subsubsection}
{3em}{0em}{1em}

\renewcommand{\maketitle}{
	\begin{center}
		{\huge\bfseries
			\theauthor}
			
		404.430.1346 $\bullet$ cacost12@asu.edu $\bullet$ US Citizen $\bullet$ linkedin.com/in/colton-acosta/
	\end{center}
}

\renewcommand{\baselinestretch}{1.0}

\begin{document}

\title{Resume}
\author{Colton Acosta}
\maketitle
\section{SUMMARY}
Junior electrical engineering student with leadership and professional experience including work in PCB design and fabrication, analog and digital electronics, embedded software, programming, and project management. Interests include avionics, control theory, signal processing, and embedded systems. Open to relocation.
\section{EDUCATION}
\textbf{B.S.E, Electrical Engineering}
\hfill 
Graduating May 2023
\linebreak
Arizona State University, Tempe, AZ 
\hfill
4.00 GPA

\section{TECHNICAL SKILLS}
\subsection{Design and Modeling}
MATLAB/Simulink, LTspice, KiCAD, Diptrace, SOLIDWORKS, Microsoft Office
\subsection{Hardware}
Microcontrollers, Soldering, Digital Multimeters, Oscilloscopes, Function Generators
\subsection{Programming} 
C, C++, Python, Linux, Git, Make, ARM Embedded Toolchain, MIPS Assembly
\section{EXPERIENCE}
\textbf{Pyramid Technologies, Inc, Mesa, AZ: Electrical Engineering Intern}
\hfill 
\vspace{0.5em}
May 2021-August 2021
\begin{itemize}
\item{Evaluated bill validation errors of a Pyramid bill acceptor in firmware us    ing an in-circuit debugger with the calibration source code, written in coldfire     assembly.}
\item{Conducted testing and qualification of replacement optocouplers including measurements of logic low voltage and slew rate for ambient temperatures ranging from 0 to 60$^{\circ}$C}
\item{Designed a test fixture IO protection PCB to protect test fixture pins from overvoltage and overcurrent conditions using schottky diodes and a PTC resettable fuse}
\item{Designed a revised USB to MDB serial converter PCB adding serial indication LEDs, signal buffering, inrush current protection, and short circuit protection to the original design}
\item{Designed a revised bill acceptor software development board by adding an electronic fuse to alleviate faulty supply/loading conditions and provide power supply fault indication}
\end{itemize}
\vspace{0.5em}
\textbf{Sun Devil Rocketry: Liquid Propulsion Team Lead}
\hfill
\vspace{0.5em}
August 2019-Present
\begin{itemize}
\item{Leading a multidisciplinary team of nine engineering students to design and develop a liquid bipropellant rocket engine}
	\item{Designing an avionics system to monitor and manage all engine hardware including including valves, transducers, thermocouples, load cells, motor drivers, transceivers, and signal processing circuitry}
\item{Designed an embedded engine controller PCB including an ARM Cortex-M4 microcontroller, a switching power supply, an embedded flash memory/micro SD card data logger, and GPIO connectors}
	\item{Programming microcontrollers with C for software control of the engine's data acquisition, actuation, flow control, and communications functionality}
	\item{Designing an instrumentation amplifier PCB with digitally programmable gain to boost available sensor outputs to measurable ranges resulting in sensor savings upwards of 10\% of club funding}
	\item{Designed and built a second order, active low-pass filter and tested the filter's noise reduction and signal reproduction by adding noise to a measured signal with a voltage summing circuit}
\item{Designing an actuation interface for the engine's AC powered flow control solenoids using solid state relays}
\end{itemize}
\section{PROJECTS}
\textbf{Flow Control Valve Actuator Control System}\hfill Fall 2020
\vspace{0.5em}
\begin{itemize}
	\item Designed and built a closed loop control system for a valve actuator for use in flow throttling applications
	\item Examined the relationship between Pulse Width Modulation duty cycle and steady state shaft speed to derive a controller output signal with a linear transfer function from controller output to shaft position  
	\item Characterized the plant transfer function with a series of step response experiments
	\item  Designed and simulated a saturated PI controller with integrator clamping using Simulink to meet performance specifications of zero steady state error of step inputs and complete rejection of step disturbances.
	\item  Built the actuator control system using a brushed DC motor, coupling shaft, Arduino controller, and quadrature rotary encoder for feedback.
\end{itemize}

\thispagestyle{empty}
\end{document}
