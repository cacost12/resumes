\documentclass{article}
\usepackage{titlesec}
\usepackage{titling}
\usepackage[margin=.5in]{geometry}
\usepackage{enumitem}

\setlist{nosep}

\titleformat{\section}
{\bfseries\normalsize}
{}
{0em}
{}[\titlerule]

\titleformat{\subsection}[runin]
{\bfseries}
{}
{0em}
{}[:]

\titleformat{\subsubsection}[runin]
{}
{$\bullet$}
{0em}
{}

\titlespacing{\subsection}
{0em}{0em}{1em}

\titlespacing{\subsubsection}
{3em}{0em}{1em}

\renewcommand{\maketitle}{
	\begin{center}
		{\huge\bfseries
			\theauthor}
			
		404.430.1346 $\bullet$ cacost12@asu.edu $\bullet$ US Citizen $\bullet$ linkedin.com/in/colton-acosta/
	\end{center}
}

\renewcommand{\baselinestretch}{1.0}

\begin{document}

\title{Resume}
\author{Colton Acosta}
\maketitle
\section{SUMMARY}
Junior electrical engineering student with leadership and collaborative project  experience including work in embedded software, programming, PCB design and fabrication, electronics, and project management. Interests include embedded firmware, controls, and signal processing. Open to relocation.
\section{EDUCATION}
\textbf{B.S.E, Electrical Engineering}
\hfill 
Graduating May 2023
\linebreak
Arizona State University, Tempe, AZ 
\hfill
4.00 GPA

\section{TECHNICAL SKILLS}
\subsection{Design and Modeling}
MATLAB/Simulink, LTspice, KiCAD, Diptrace, SOLIDWORKS, Microsoft Office
\subsection{Hardware}
Microcontrollers, Soldering, Digital Multimeters, Oscilloscopes, Function Generators
\subsection{Programming} 
C, C++, Python, Linux, Git, ARM Embedded Toolchain, MIPS Assembly
\section{EXPERIENCE}
\textbf{Pyramid Technologies, Inc, Mesa, AZ: Electrical Engineering Intern}
\hfill 
\vspace{0.5em}
May 2021-August 2021
\begin{itemize}
\item{Evaluated bill validation errors of a Pyramid bill acceptor in firmware using an in-circuit debugger with the calibration source code, written in coldfire assembly.}
\item{Conducted phototransistor data collection and analysis with over 150 LEDs to find a viable replacement bill validation LED that would work at scale production with minimal firmware modifications}
\item{Designed a revised bill acceptor software development board by adding an electronic fuse to alleviate faulty supply/loading conditions and provide power supply fault indication}
\item{Designed a revised USB to MDB serial converter PCB adding serial indication LEDs, signal buffering, inrush current protection, and short circuit protection to the original design}
\item{Wrote Python scripts to calculate external component design values from input specifications and datasheet guidelines}
\end{itemize}
\vspace{0.5em}
\textbf{Sun Devil Rocketry: Liquid Propulsion Team Lead}
\hfill
\vspace{0.5em}
August 2019-Present
\begin{itemize}
\item{Leading a multidisciplinary team of nine engineering students to design and develop a liquid bipropellant rocket engine}
	\item{Designing an avionics system to monitor and manage all engine hardware including including valves, transducers, thermocouples, load cells, motor drivers, transceivers, and signal processing circuitry}
\item{Designed an embedded engine controller PCB including an ARM Cortex-M4 microcontroller, a switching power supply, an embedded flash memory/micro SD card data logger, and GPIO connectors}
	\item{Programming the engine controller with C for software control of the engine's data acquisition, actuation, flow control, and communications functionality}
	\item{Wrote a C++ program to generate Gaussian noise for hardware filter testing by writing an algorithm for computing values of an inverse Gaussian cumulative distribution function}
	\item{Wrote a C program to encode the state of the engine's valves using bit operators for efficient serial data transmission}
	\item{Interfacing temperature, pressure, thrust, and flow measurements with a Python graphical user interface}
	\item{Wrote, compiled, and debugged all C and C++ code using Linux command line tools such as gcc, g++, gdb, and vim}
	\item{Documented project progress in published AIAA Propulsion and Energy conference paper}
\end{itemize}
\section{PROJECTS}
\vspace{0.5em}
\textbf{Flow Control Valve Actuator Control System}\hfill Fall 2020
\vspace{0.5em}
\begin{itemize}
	\item Designed and built a closed loop control system for a valve actuator for use in flow throttling applications
	\item Examined the relationship between Pulse Width Modulation duty cycle and steady state shaft speed to derive a controller output signal with a linear transfer function from controller output to shaft position  
	\item Characterized the plant transfer function with a series of step response experiments
	\item  Implemented a saturated PI controller with integrator clamping in C++, and simulated the performance using Simulink to meet specifications of zero steady state error of step inputs and complete rejection of step disturbances.
	\item  Built the actuator control system using a brushed DC motor, coupling shaft, Arduino controller, and quadrature rotary encoder for feedback.
\end{itemize}
\vspace{0.5em}
\thispagestyle{empty}
\end{document}
