\documentclass{article}
\usepackage{titlesec}
\usepackage{titling}
\usepackage[margin=.5in]{geometry}
\usepackage{enumitem}

\setlist{nosep}

\titleformat{\section}
{\bfseries\normalsize}
{}
{0em}
{}[\titlerule]

\titleformat{\subsection}[runin]
{\bfseries}
{}
{0em}
{}[:]

\titleformat{\subsubsection}[runin]
{}
{$\bullet$}
{0em}
{}

\titlespacing{\subsection}
{0em}{0em}{1em}

\titlespacing{\subsubsection}
{3em}{0em}{1em}

\renewcommand{\maketitle}{
	\begin{center}
		{\huge\bfseries
			\theauthor}
			
		404.430.1346 $\bullet$ cacost12@asu.edu $\bullet$ US Citizen $\bullet$ coltonacosta.com $\bullet$ linkedin.com/in/colton-acosta/
	\end{center}
}

\renewcommand{\baselinestretch}{1.0}

\begin{document}

\title{Resume}
\author{Colton Acosta}
\maketitle
\section{EDUCATION}
\textbf{B.S.E, Electrical Engineering}
\hfill 
May 2023
\linebreak
Arizona State University, Tempe, AZ 
\hfill
4.00 GPA

\section{TECHNICAL SKILLS}
\subsection{Software} 
C, C++, Python, Assembly, Linux, Git, Make, ARM, Visual Studio 
\subsection{Hardware}
Verilog, Microcontrollers, FPGA, Soldering (SMD), Multimeters, Oscilloscopes, Function Generators 
\subsection{Design/Modeling}
LTspice, KiCAD, DipTrace, MATLAB/Simulink, Cadence, SolidWorks
\section{EXPERIENCE}
\textbf{Undergraduate Research Assistant: SCALE Advanced CMOS}
\hfill
\vspace{0.5em}
January 2022--Present
\begin{itemize}
\item{Assisted in current-voltage and random telegraph noise data collection of CMOS devices to be used in compact models}
\item{Designed over ten PCBs for mounting test devices and interfacing with semiconductor parameter analyzers}
\item{Constructed a CMOS measurement setup rated for cryogenic temperatures to emulate temperature conditions in space}
\item{Designed and simulated a transimpedance amplifier to amplify CMOS drain currents and filter high frequency noise}
\end{itemize}
\vspace{1em}
\textbf{Garmin Aviation: Embedded Software Engineering Intern}
\hfill
\vspace{0.5em}
May 2022--August 2022
\begin{itemize}
\item{Developed certification software for a new Vulkan graphics driver to be used in safety-critical avionics systems}
\item{Wrote unit tests with randomized test vectors in C to test the GPU driver source code with maximal coverage}
\item{Debugged compiler errors of ARM and Windows builds using Visual Studio and MSBuild XML schemas}
\item{Resolved runtime errors caused by randomized test vectors by analyzing the source code functions and manually setting up data structures, pointers, arrays, and buffers}
\end{itemize}
\vspace{1em}
\textbf{Sun Devil Rocketry: President and Avionics Team Founder}
\hfill
\vspace{0.5em}
August 2021--May 2022
\begin{itemize}
\item{Oversaw all activities of a technical student organization with three rocket propulsion teams, two amateur rocketry teams, a K-12 outreach program, and over 50 members   }
\item{Facilitated all project development by holding meetings and design reviews, writing budget proposals, organizing launch logistics, mentoring, and maintaining industry/university relations}
\item{Founded a new avionics team to design the club's first flight computer and promote the development of electrical and software engineering skills among students interested in the aerospace industry}
\vspace{1em}
\end{itemize}
\textbf{Pyramid Technologies, Inc, Mesa, AZ: Electrical Engineering Intern}
\hfill 
\vspace{0.5em}
May 2021--August 2021
\begin{itemize}
\item{Evaluated bill validation errors of a bill acceptor's firmware using an in-circuit debugger and assembly source code}
\item{Revised a switching power supply and serial opto-isolator PCB to be usable with multiple bill acceptors}
\item{Qualified new optocouplers by measuring logic levels and slew rate for ambient temperatures ranging from 0 to 60$^{\circ}$C}
\item{Designed a new PCB to protect test fixture pins from overvoltage and overcurrent conditions
	using schottky diodes and a PTC resettable fuse}
\item{Collected and analyzed phototransistor data on over 150 LEDs to find a viable bill validation
	LED that would work at scale production without firmware modifications}
\item{Added serial indication LEDs, signal buffering, inrush current protection, and short circuit protection to a USB to MDB serial interface PCB}
\item{Designed a revised bill acceptor software development board by adding an electronic fuse to alleviate faulty sup-
	ply/loading conditions and provide power supply fault indication}
\item{Performed DC load testing on a new 120V AC power supply to measure power trace voltage drops at full load}
\item{Conducted electrical tests and wrote qualification documents for replacement PCB parts to resolve procurement issues}
\item{Tracked project progress and managed feedback on PCB designs and layouts with git and bitbucket}
\item{Resolved electrical issues with dysfunctional test fixtures and equipment used by engineers and production staff}
\item{Wrote Python scripts to calculate external component design values from input specifications and datasheet guidelines}
\end{itemize}
\section{PROJECTS}
\textbf{Sun Devil Rocketry: Flight Computer}
\hfill
\vspace{0.5em}
January 2021--Present
\begin{itemize}
\item{Developing a flight computer for high-powered rockets to implement recovery, control, and telemetry functionality}
\item{Equipped the embedded computer with an ARM Cortex-M7 microcontroller, a 9-axis IMU, GPS, a LoRa wireless module, a micro SD card, external flash, and a USB interface}
\item{Programmed the computer with C for low level control of the microcontroller's UART, I2C, SPI, and GPIO peripherals}
\item{Developed APIs for accessing IMU and barometric pressure sensor data from application code}
\item{Wrote a data-logger application to collect real flight data for testing, and successfully recovered flight data from a Sun Devil Rocketry high power launch}
\end{itemize}
\vspace{1em}
\textbf{Sun Devil Rocketry: Engine Controller}
\hfill
\vspace{0.5em}
August 2019--Present
\begin{itemize}
\item{Developing a controller for a liquid rocket engine to manage engine hardware and automate ignition sequencing}
\item{Designed the PCB using an ARM Cortex-M7 microcontroller, a switching power supply, external flash, an SD card, ignition terminals, sensor peripherals, a USB interface, and a wireless command and control interface}
\item{Developed ignition and data-logging APIs in C to abstract low-level hardware control functionality}
\item{Programmed a Python interface for real-time visualization of temperature, pressure, thrust, and flow measurements}
\item{Amplified pressure transducer differential outputs to measurable ranges using a programmable amplifier circuit in order to save upwards of 10\% of club funding in new sensor costs}
\end{itemize}
\vspace{1em}
\textbf{Sun Devil Rocketry: Valve Controller}
\hfill
\vspace{0.5em}
Spring 2022
\begin{itemize}
\item{Designed, built, and tested a controller to actuate rocket engine valves using an ARM Cortex-M7 microcontroller, solid state relays, a pulse interface, and motor sensors.}
\item{Calibrated valve shaft initial positions using an optoelectronic photogate sensor with customized form factor}
\item{Designed an optically-isolated voltage monitoring circuit to alert the controller when solenoid power is lost}
\item{Programmed the controller in C to process valve actuation commands from the main engine controller}
\item{Developed a solenoid control API in C to implement basic solenoid actuation functions to simplify the application code}
\end{itemize}
\vspace{1em}
\textbf{Flow Control Valve Actuator Control System}\hfill Fall 2020
\vspace{0.5em}
\begin{itemize}
	\item Designed and built a closed loop control system for a valve actuator for use in flow throttling applications
	\item Examined the relationship between Pulse Width Modulation duty cycle and steady state shaft speed to derive a controller output signal with a linear transfer function from controller output to shaft position  
	\item Characterized the plant transfer function with a series of step response experiments
	\item  Implemented a saturated PI controller with integrator clamping in C++, and simulated the performance using Simulink to meet specifications of zero steady state error of step inputs and complete rejection of step disturbances
	\item  Built the actuator using a brushed DC motor, coupling shaft, Arduino, and quadrature rotary encoder for feedback
\end{itemize}
\vspace{0.5em}
\thispagestyle{empty}
\thispagestyle{empty}
\end{document}
